\documentclass[a4paper]{article}
\usepackage[T1]{fontenc}
\usepackage[ngerman]{babel}
\usepackage{array}
\usepackage{booktabs}
\usepackage[margin=1in]{geometry}


\begin{document}
\section{Tabellen}

% Typ	Beschreibung
% l	linksbündige Spalte
% c	zentrierte Spalte
% r	rechtsbündige Spalte
% p{breite}	eine Spalte der festen Breite breite; der Text wird automatisch umgebrochen und im Blocksatz ausgegeben
% m{breite}	wie p, allerdings mit vertikaler Zentrierung zur übrigen Zeile
% b{breite}	wie p, aber an der Unterkante ausgerichtet
% w{ausrichtung}{breite}	gibt den Inhalt der festen Breite breite aus, bei zu großem Inhalt überlappt der Inhalt die Spaltenbreite. Die horizontale Ausrichtung kann aus l, c und r ausgewählt werden.
% W{ausrichtung}{breite}	wie w, es wird allerdings eine Warnung ausgegeben, wenn der Inhalt breiter als breite sein sollte.

\begin{tabular}{lll}
    Tier   & Nahrung & Größe  \\
    Hund   & Fleisch & mittel \\
    Pferd  & Heu     & groß   \\
    Frosch & Fliegen & klein  \\
\end{tabular}

\section{Lange Tabellen mit Festbreite}

\begin{tabular}{cp{9cm}}
    Tier  & Beschreibung                                                         \\
    Hund  & Der Hund ist ein Mitglied der Gattung Canis, welche Teil der Familie
    Canidae ist, und das weitverbreitetste Landraubtier.                         \\
    Katze & Katzen sind eine domestizierte Art kleiner fleischfressender
    Säugetiere. Sie ist die einzige domestizierte Art der Familie Felidae
    und wird häufig als Hauskatze bezeichnet, um sie von den wildlebenden
    Mitglieder dieser Familie abzugrenzen.                                       \\
\end{tabular}

\section{Wildcard-Präambel}

\begin{tabular}{*{3}{l}}
    Tier   & Nahrung & Größe  \\
    Hund   & Fleisch & mittel \\
    Pferd  & Heu     & groß   \\
    Frosch & Fliegen & klein  \\
\end{tabular}

\section{Linien}
\begin{tabular}{lll}
    \toprule
    Tier   & Nahrung & Größe  \\
    \midrule
    Hund   & Fleisch & mittel \\
    Pferd  & Heu     & groß   \\
    Frosch & Fliegen & klein  \\
    \bottomrule
\end{tabular}

\section{Cmidrule}

\begin{tabular}{lll}
    \toprule
    Tier   & Nahrung & Größe  \\
    \midrule
    Hund   & Fleisch & mittel \\
    \cmidrule{1-1}
    Pferd  & Heu     & groß   \\
    \cmidrule{1-1}
    \cmidrule{3-3}
    Frosch & Fliegen & klein  \\
    \bottomrule
\end{tabular}

\section{Cmidrule - shortened}

\begin{tabular}{lll}
    \toprule
    Tier   & Nahrung & Größe  \\
    \midrule
    Hund   & Fleisch & mittel \\
    \cmidrule(r){1-2}
    Pferd  & Heu     & groß   \\
    \cmidrule(r){1-1}
    \cmidrule(rl){2-2}
    \cmidrule(l){3-3}
    Frosch & Fliegen & klein  \\
    \bottomrule
\end{tabular}

\section{Addlinespace}

\begin{tabular}{cp{9cm}}
    \toprule
    Tier  & Beschreibung                                                         \\
    \midrule
    Hund  & Der Hund ist ein Mitglied der Gattung Canis, welche Teil der Familie
    Canidae ist, und das weitverbreitetste Landraubtier.                         \\
    \addlinespace
    Katze & Katzen sind eine domestizierte Art kleiner fleischfressender
    Säugetiere. Sie ist die einzige domestizierte Art der Familie Felidae
    und wird häufig als Hauskatze bezeichnet, um sie von den wildlebenden
    Mitglieder dieser Familie abzugrenzen.                                       \\
    \bottomrule
\end{tabular}

\section{Zellen zusammenfügen}

\subsection{Horizontal}

\begin{tabular}{lll}
    \toprule
    Tier         & Nahrung                       & Größe  \\
    \midrule
    Hund         & Fleisch                       & mittel \\
    Pferd        & Heu                           & groß   \\
    Frosch       & Fliegen                       & klein  \\
    Wolpertinger & \multicolumn{2}{c}{unbekannt}          \\
    \bottomrule
\end{tabular}

\subsection{Sonderfall: Eine Spalte}

\begin{tabular}{lll}
    \toprule
    \multicolumn{1}{c}{Tier} & \multicolumn{1}{c}{Nahrung}   & \multicolumn{1}{c}{Größe} \\
    \midrule
    Hund                     & Fleisch                       & mittel                    \\
    Pferd                    & Heu                           & groß                      \\
    Frosch                   & Fliegen                       & klein                     \\
    Wolpertinger             & \multicolumn{2}{c}{unbekannt}                             \\
    \bottomrule
\end{tabular}

\subsection{Vertikal}

Vertikales zusammenfügen ist in \LaTeX{} nicht erlaubt, aber man kann tricksen:

\begin{tabular}{lll}
    \toprule
    Gruppe          & Tier    & Größe  \\
    \midrule
    Pflanzenfresser & Pferd   & groß   \\
                    & Reh     & mittel \\
                    & Hase    & klein  \\
    \addlinespace
    Fleischfresser  & Hund    & mittel \\
                    & Katze   & klein  \\
                    & Löwe    & groß   \\
    \addlinespace
    Allesfresser    & Krähe   & klein  \\
                    & Bär     & groß   \\
                    & Schwein & mittel \\
    \bottomrule
\end{tabular}

\end{document}