\documentclass{article}
\usepackage[T1]{fontenc}
\usepackage[ngerman]{babel}

\newcommand{\diff}{\mathop{}\!d}            % für kursiv
% \newcommand{\diff}{\mathop{}\!\mathrm{d}} % für aufrecht

\begin{document}
Ein Satz mit Mathematik im Text: \(y = mx + c\).
Ein zweiter Satz mit Mathematik in der Zeile: \(5^{2}=3^{2}+4^{2}\).

Ein zweiter Absatz mit alleinstehender Gleichung.
\[
    y = mx + c
\]
Man beachte, dass der Absatz nach der Gleichung fortgeführt wird.

Hochgestellter Exponent \(a^{b}\) und tiefgestellter Index \(a_{b}\). Spezielle Mathematik ist auch möglich: \(y = 2 \sin \theta^{2}\).

Ein Absatz über eine größere Gleichung
\[
    \int_{-\infty}^{+\infty} e^{-x^2} \, dx
\]

Ein Absatz über eine größere Gleichung mit eigenem Kommando
\[
    \int_{-\infty}^{+\infty} e^{-x^2} \diff x
\]

\section{Nummerierte Gleichungen}

\begin{equation}
    \int_{-\infty}^{+\infty} e^{-x^2} \diff x
\end{equation}
\end{document}